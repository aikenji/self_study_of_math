\section{Natural Numbers}

\subsection{Peano Postulates}

\begin{definition}{}{}
    Let $S : N \to N$ and let $A \subseteq N$. Then $A$ is said to be closed under $S$ iff
    \begin{equation*}
        S(x) \in A \text{ for all } x \in A
    \end{equation*}
\end{definition}

\begin{definition}{Peano System}{}
    Let $(N,S,e)$ be an ordered triple that consists of a set $N$, a
    function $S : N \to N$, and an element $e \in N$. Then $(N,S,e)$
    is a \textbf{Peano system} if the following conditions hold:
    \begin{enumerate}

        \item $e \notin ranS$
        \item $S$ is injective.
        \item For all $A \subseteq N$
        \begin{equation*}
            e \in A \land A \text{ is closed under } S \implies A = N
        \end{equation*}

    \end{enumerate}
    
\end{definition}

\subsection{Inductive Sets}

In this section, we will construct natural numbers under the 
architecture of set theory.

\begin{definition}{Successor}{}
    For each set $x$, the \textbf{successor} $x^{+}$ is the set defined by 
    \begin{equation*}
        x^{+} = x \cup \{x\}
    \end{equation*}
\end{definition}

\begin{proposition}{}{}
    \begin{enumerate}

        \item $a \in x^{+} \iff a \in x \lor a = x$
        \item $x \in x^{+}$
        \item $x \subseteq x^{+}$

    \end{enumerate}
\end{proposition}

\begin{example}
   The first few natural numbers as follows
   \begin{itemize}

       \item $0 = \varnothing$
       \item $1 = \{0\}$
       \item $2 = \{0,1\}$
       \item $3 = \{0,1,2\}$
       \item $4 = \{0,1,2,3\}$

   \end{itemize} 
\end{example}


