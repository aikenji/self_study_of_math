\section{Relations and Functions}
\subsection{Ordered pairs}
In mathematics, relations and functions are usually defined in terms of ordered pairs. The key property the ordered pairs satisfied is as below
\begin{definition}{Ordered pairs}{}
    Basic property the ordered pairs should satisfy, which is also the axiomatic definition of ordered pairs
    \begin{equation*}
        \langle a,b \rangle = \langle c,d \rangle \iff a = b \land c = d
    \end{equation*}
\end{definition}
In 1921, Kazimierz Kuratowski give the construction of ordered pair based on ZF systems. 
\begin{definition}{Set based construction of ordered pairs}{}
    The \textbf{ordered pair}, with first component x and second component y, is the set defined by
    \begin{equation*}
        \langle x,y \rangle = \{\{x\}, \{x, y\}\}
    \end{equation*}
\end{definition}

\begin{lemma}{}{}
    Let $A$, $B$ be sets, there exists a unique set $\mathcal{D}$ s.t. for all $X$,
    \begin{equation*}
        X \in \mathcal{D} \iff X = \langle x,y \rangle \ \text{for some}\ x \in A \land y \in B.
    \end{equation*}
    or,
    \begin{equation*}
        X \in \mathcal{D} \iff (\exists x \in A)(\exists y \in B)(X = \langle x,y \rangle)
    \end{equation*}
    We shall denote $\mathcal{D}$ by $\{\langle x,y \rangle : x \in A \land y \in B\}$.
\end{lemma}

\begin{proof} 
    $X = \langle x,y \rangle \ \text{for some}\ x \in A \land y \in B \implies X = \{\{x\},\{x,y\}\} \in \mathcal{P}(\mathcal{P}(A \cup B) )$.\\
    The set $\mathcal{D}$ is uniquely defined by \Cref{class}.
\end{proof}

\begin{definition}{Cartesian product}{}
    Given sets $A$ and $B$, the \textbf{Cartesian product} $A \times B$ is defined to be the set
    \begin{equation*}
        A \times B = \{\langle x,y \rangle : x \in A \land y \in B\}
    \end{equation*}
    which is uniquely given by the above lemma.
\end{definition}

\subsection{Relations}
\begin{definition}{Relations}{}
    A \textbf{Relation} $R$ is a set of orded pairs. In other words, for all $x$,
    \begin{equation*}
        x \in R \iff x = \langle a,b \rangle \ \text{for some}\ a\ \text{and}\ b.
    \end{equation*}
\end{definition}

\begin{definition}{}{}
    Let $A$ and $B$ be sets. 
    \begin{enumerate}

        \item A \textbf{relation from $A$ to $B$} is a subset of $A \times B$, or $R \subseteq A \times B$. 
        \item A \textbf{relation on $A$} is a subset of $A \times A$, or $R \subseteq A \times A$. 
    \end{enumerate}
\end{definition}

\begin{remarks}
    by subset axiom, given a formula $\varphi(x,y)$, and sets $A$ and $B$, we can construct a relation R,
    \begin{equation*}
        R = \{\langle x,y \rangle \in A \times B : \varphi(x,y)\}
    \end{equation*}
\end{remarks}

\begin{lemma}{}{}
    Let $R$ be a relation and $A = \cup \cup R$. Then $R \subseteq A \times A$.
\end{lemma}

We conclude that every relation can be viewed as a relation on one set.

\begin{definition}{}{}
    Let R is a relation:
    \begin{enumerate}

        \item The \textbf{domain} of $R$: $dom{R} = \{x : \exists y (\langle x,y \rangle \in R)\}$.
        \item The \textbf{range} of $R$: $ran{R} = \{y : \exists x (\langle x,y \rangle \in R)\}$.

    \end{enumerate}
\end{definition}

\begin{remarks}
    we have concluded that $x, y \in \cup \cup R$, hence the domain and range of relation $R$ are proofed to to be sets by \Cref{class}.
\end{remarks}

\begin{definition}{Operations on relations}{}
    Let $R$ and $S$ be relations, and given set $A$
    \begin{enumerate}

        \item the \textbf{inverse} of $R$ is the relation
        \begin{equation*}
            R^{-1} = \{\langle y,x \rangle : \langle x,y \rangle \in R\}
        \end{equation*}
        or
        \begin{equation*}
            \langle y,x \rangle \in R^{-1} \iff \langle x,y \rangle \in R
        \end{equation*}
        \item the \textbf{restriction} of $R$ to $A$ is the relation
        \begin{equation*}
            R \vert_{A} = \{\langle x,y \rangle : \langle x,y \rangle \in R \land x \in A\}
        \end{equation*}
        \item the \textbf{image} of $A$ under R is the set
        \begin{equation*}
            R[A] = \{y : \exists x \in A (\langle x,y \rangle \in R)\}
        \end{equation*}
        \item the \textbf{inverse image} of A under R is the set
        \begin{equation*}
            R^{-1}[A] = \{x : \exists y \in A(\langle x,y \rangle \in R)\}
        \end{equation*}
        \item the \textbf{composition} of $R$ and $S$ is the relation
        \begin{equation*}
            R \circ S = \{\langle x,z \rangle : \exists y (\langle x,y \rangle \in S \land \langle y,z \rangle \in R)\}
        \end{equation*}
    \end{enumerate}
\end{definition}

\begin{proposition}{}{}
    Let $R$,$S$ and $T$ be relations, then
    \begin{enumerate}

        \item $dom R^{-1} = ran R$
        \item $ran R^{-1} = dom R$
        \item $(R^{-1})^{-1} = R$ 
        \item $(R \circ S)^{-1} = S^{-1} \circ R^{-1}$
        \item $R \circ (S \circ T) = (R \circ S) \circ T$

    \end{enumerate}
\end{proposition}

\begin{proposition}{}{}
    Suppose $\mathcal{F}$ is a collections of sets, and R is a relation
    \begin{enumerate}

        \item $R[\bigcup \mathcal{F}] = \bigcup \{R[F] : F \in \mathcal{F}\}$
        \item $R[\bigcap \mathcal{F}] \subseteq \bigcap \{R[F] : F \in \mathcal{F}\}$ ($\mathcal{F}$ is nonempty)

    \end{enumerate}
\end{proposition}
The above proposition proclaim that the image of a union is the union of the images,” whereas “the image of an intersection is a subset of the intersection of the images.

\begin{definition}{Single-rooted}{}
    A relation $R$ is \textbf{single-rooted} if for every $y \in ran R$, there is exactly one $x$ s.t. $\langle x,y \rangle \in R$. In other words
    \begin{equation*}
        \langle x,z \rangle \in R \land \langle y,z \rangle \in R \implies x = y
    \end{equation*}
\end{definition}

\begin{corollary}{}{}
    Let $R$ be a single-rooted relation. Suppose that A and B are sets and $\mathcal{F}$ is a nonempty collection.
    \begin{enumerate}

        \item $R[\bigcap \mathcal{F}] = \bigcap \{R[F] : F \in \mathcal{F}\}$ 
        \item $R[A \setminus B] = R[A] \setminus R[B]$

    \end{enumerate}
\end{corollary}

\subsubsection{Equivalence relations}
Let $R$ be a relation. If $\langle x,y \rangle \in R$, we shall say that x is related to y, sometimes it is also written as $xRy$. 

\begin{definition}{Equivalence relation}{}
    Let $\sim$ be a relation on $A$. For any $x$, $y$ and $z$ in $A$.
    \begin{enumerate}

        \item \textbf{reflexive}: $x \sim x$
        \item \textbf{symmetric}: $x \sim y \implies y \sim x$
        \item \textbf{transitive}: $x \sim y \land y \sim x \implies x \sim z$

    \end{enumerate}
    A relation is called a \textbf{equivalence relation} on A, if it is reflexive, symmetric and transitive.
\end{definition}

\begin{remarks}
    \begin{enumerate}

        \item The relation is \textit{reflexive} if every element in the set A is related to itself.
        \item The relation is \textit{symmetric} if whenever x is related to y, then y is related to x.
        \item The relation is \textit{transitive} if whenever x is related to y and y is related to z, then x is also related to z.

    \end{enumerate}
\end{remarks}

\begin{definition}{Partition}{}
    Let $A$ be a set. Let $\mathcal{P}$ be a collection of nonempty subsets of $A$. We say that $\mathcal{P}$ is a \textbf{partition} of $A$ if the following two conditions hold:
    \begin{enumerate}

        \item for every $x \in A$, there is a $P \in \mathcal{P}$ s.t. $x \in P$.
        \item for any $P$ and $Q$ in $\mathcal{P}$, if $P \neq Q$, then $P \cap Q = \emptyset$. Sometimes we call that sets in $\mathcal{P}$ are \textbf{pairwise disjoint}.    

    \end{enumerate}
\end{definition}
    
\begin{definition}{Equivalence class}{}
    Let $\sim$ be an equivalence relation on $A$, and $a$ be an element of $A$. The \textbf{equivalence class} of $a$ is the set consists of all elements of $A$ which are related to $a$. We denote it as
    \begin{equation*}
        [a] = \{x \in A: x \sim a\}
    \end{equation*}
\end{definition}

\begin{theorem}{}{}
    Let $\sim$ be an equivalence relation of $A$. For any $a$, $b \in A$,
    \begin{equation*}
        a \sim b \iff [a] = [b]
    \end{equation*}
\end{theorem}

\begin{proof}
    (\implies )
    \begin{align*}
        x \in [a] &\implies x \sim a\\
        &\implies x \sim a \land a \sim b \implies x \sim b\\
        &\implies x \in [b]\\
    \end{align*}
    \begin{align*}
        x \in [b] &\implies x \sim b\\
        &\implies x \sim b \land b \sim a \implies x \sim a\\
        &\implies x \in [a]\\
        &\implies [a] = [b]\\
    \end{align*}
    ($\Leftarrow $ )
    \begin{align*}
        [a] = [b] &\implies a \in [b]\\
        &\implies a \sim b
    \end{align*}
\end{proof}

\begin{corollary}{}{}
    Let $\sim$ be an equivalence relation of $A$. For any $a$, $b \in A$,
    \begin{equation*}
        a \in [b] \iff [a] = [b]
    \end{equation*}
\end{corollary}

\begin{theorem}{Fundamental Theorem of Equivalence Relation 1}{}
    Let $\sim$ be an equivalence relation on $A$. Then the collection $A/\sim = \{[a]: a \in A\} $ is a partition of A. Sometimes we call $A/\sim$ the \textbf{quotient set induced by} $\sim$.
\end{theorem}

\begin{proof}
    Firstly, we proof the existence of the collection $A/\sim$. Obviously, The collection can be written as $\{X: \exists a \in A (X = [a])\}$, which implies that $X \in \mathcal{P}(A) $. By \Cref{class}, $A/\sim$ is a set.\\
    Then we proof $A/\sim$ is a partition of A.
    \begin{enumerate}

        \item for every $a \in A$, we can find $[a] \in A/\sim$ s.t. $a \in [a]$.
        \item Let $[a]$, $[b] \in A/\sim$, and  $[a] \neq [b]$. Assume $[a] \cap [b] \neq \emptyset\\$
        \begin{align*}
            &[a] \cap [b] \neq \emptyset \\
            &\implies x \in [a] \land x \in [b]\\
            &\implies [x] = [a] \land [x] = [b]\\
            &\implies [a] = [b] 
        \end{align*} 

        Which is a contradiction.

    \end{enumerate}
\end{proof}

\begin{theorem}{Fundamental Theorem of Equivalence Relation 2}{}
    Let $\mathcal{P}$ be a partition of a set $A$. Then there is an equivalence relation $\sim$ on $A$ defined as follows: 
    \begin{equation*}
        x \sim y \iff x \in P \land y \in P \ \text{for some}\ P \in \mathcal{P}.
    \end{equation*}
\end{theorem}

\begin{proof}
    We just proof the relation $\sim$ satisfy reflexive, symmetric and transitive properties.
    \begin{enumerate}

        \item reflexive:
        \begin{equation*}
            \forall x \in A \exists P \in \mathcal{P} (x \in P) \implies \forall x \in A (x \sim x)
        \end{equation*}
        \item symmetric: trivial.
        \item transitive:
        \begin{align*}
            &x \sim y \land y \sim z\\
            &\implies (x,y \in P \ \text{for some}\ P \in \mathcal{P})\land (y,z \in S \ \text{for some}\ S \in \mathcal{P})\\
            &\implies y \in P \land y \in S\\
            &\implies P = S\\
            &\implies x,z \in P \ \text{for some}\ P \in \mathcal{P}\\
            &\implies x \sim z
        \end{align*}

    \end{enumerate}
\end{proof}

\begin{examples}
    There is an equivalence relation on $\mathbb{Z}$, defined by
    \begin{equation*}
        m \sim n \iff 3 | (m - n)
    \end{equation*}
    We get the equivalence class $[n]$:
    \begin{equation*}
        [n] = \{m \in \mathbb{Z}: m \sim n\} = \{3k+n: k \in \mathbb{Z}\}
    \end{equation*}
    \begin{align*}
        [0] &= \{...-3,0,3,6\}\\
        [1] &= \{...-2,1,4,7\}\\
        [2] &= \{...-1,2,5,8\}\\
    \end{align*}
    Finally, we can the quotient set:
    \begin{equation*}
        \mathbb{Z}/\sim = \{[0],[1],[2]\}
    \end{equation*}
\end{examples}

\subsubsection{Order relations}

