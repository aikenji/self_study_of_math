\section{Relations and Functions}
\subsection{Ordered pairs}
In mathematics, relations and functions are usually defined in terms of ordered pairs. The key property the ordered pairs satisfied is as below
\begin{definition}{Ordered pairs}{}
    Basic property the ordered pairs should satisfy, which is also the axiomatic definition of ordered pairs
    \begin{equation*}
        \langle a,b \rangle = \langle c,d \rangle \iff a = b \land c = d
    \end{equation*}
\end{definition}
In 1921, Kazimierz Kuratowski give the construction of ordered pair based on ZF systems. 
\begin{definition}{Set based construction of ordered pairs}{}
    The \textbf{ordered pair}, with first component x and second component y, is the set defined by
    \begin{equation*}
        \langle x,y \rangle = \{\{x\}, \{x, y\}\}
    \end{equation*}
\end{definition}

\begin{lemma}{}{}
    Let $A$, $B$ be sets, there exists a unique set $\mathcal{D}$ s.t. for all $X$,
    \begin{equation*}
        X \in \mathcal{D} \iff X = \langle x,y \rangle \ \text{for some}\ x \in A \land y \in B.
    \end{equation*}
    or,
    \begin{equation*}
        X \in \mathcal{D} \iff (\exists x \in A)(\exists y \in B)(X = \langle x,y \rangle)
    \end{equation*}
    We shall denote $\mathcal{D}$ by $\{\langle x,y \rangle : x \in A \land y \in B\}$.
\end{lemma}

\begin{proof} 
    $X = \langle x,y \rangle \ \text{for some}\ x \in A \land y \in B \implies X = \{\{x\},\{x,y\}\} \in \mathcal{P}(\mathcal{P}(A \cup B) )$.\\
    The set $\mathcal{D}$ is uniquely defined by \Cref{class}.
\end{proof}

\begin{definition}{Cartesian product}{}
    Given sets $A$ and $B$, the \textbf{Cartesian product} $A \times B$ is defined to be the set
    \begin{equation*}
        A \times B = \{\langle x,y \rangle : x \in A \land y \in B\}
    \end{equation*}
    which is uniquely given by the above lemma.
\end{definition}

\subsection{Relations}
\begin{definition}{Relations}{}
    A \textbf{Relation} $R$ is a set of orded pairs. In other words, for all $x$,
    \begin{equation*}
        x \in R \iff x = \langle a,b \rangle \ \text{for some}\ a\ \text{and}\ b.
    \end{equation*}
\end{definition}

\begin{definition}{}{}
    Let $A$ and $B$ be sets. 
    \begin{enumerate}

        \item A \textbf{relation from $A$ to $B$} is a subset of $A \times B$, or $R \subseteq A \times B$. 
        \item A \textbf{relation on $A$} is a subset of $A \times A$, or $R \subseteq A \times A$. 
    \end{enumerate}
\end{definition}

\begin{remarks}
    by subset axiom, given a formula $\varphi(x,y)$, and sets $A$ and $B$, we can construct a relation R,
    \begin{equation*}
        R = \{\langle x,y \rangle \in A \times B : \varphi(x,y)\}
    \end{equation*}
\end{remarks}

\begin{lemma}{}{}
    Let $R$ be a relation and $A = \cup \cup R$. Then $R \subseteq A \times A$.
\end{lemma}

We conclude that every relation can be viewed as a relation on one set.

\begin{definition}{}{}
    Let R is a relation:
    \begin{enumerate}

        \item The \textbf{domain} of $R$: $dom{R} = \{x : \exists y (\langle x,y \rangle \in R)\}$.
        \item The \textbf{range} of $R$: $ran{R} = \{y : \exists x (\langle x,y \rangle \in R)\}$.

    \end{enumerate}
\end{definition}

\begin{remarks}
    we have concluded that $x, y \in \cup \cup R$, hence the domain and range of relation $R$ are proofed to to be sets by \Cref{class}.
\end{remarks}

\begin{definition}{Operations on relations}{}
    Let $R$ and $S$ be relations, and given set $A$
    \begin{enumerate}

        \item the \textbf{inverse} of $R$ is the relation
        \begin{equation*}
            R^{-1} = \{\langle y,x \rangle : \langle x,y \rangle \in R\}
        \end{equation*}
        or
        \begin{equation*}
            \langle y,x \rangle \in R^{-1} \iff \langle x,y \rangle \in R
        \end{equation*}
        \item the \textbf{restriction} of $R$ to $A$ is the relation
        \begin{equation*}
            R \vert_{A} = \{\langle x,y \rangle : \langle x,y \rangle \in R \land x \in A\}
        \end{equation*}
        \item the \textbf{image} of $A$ under R is the set
        \begin{equation*}
            R[A] = \{y : \exists x \in A (\langle x,y \rangle \in R)\}
        \end{equation*}
        \item the \textbf{inverse image} of A under R is the set
        \begin{equation*}
            R^{-1}[A] = \{x : \exists y \in A(\langle x,y \rangle \in R)\}
        \end{equation*}

    \end{enumerate}
    
\end{definition}
